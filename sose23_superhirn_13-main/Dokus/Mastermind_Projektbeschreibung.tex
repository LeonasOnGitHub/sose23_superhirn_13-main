\documentclass[10pt]{article}
\usepackage[usenames]{color} % Farbunterstützung
\usepackage{amssymb}	% Mathe
\usepackage{amsmath} % Mathe
\usepackage[utf8]{inputenc} % Direkte Eingabe von Umlauten und anderen Diakritika
\title{Mastermind}
\author{Team 13}
\date{\today}
\begin{document}
\maketitle

\tableofcontents

\section{Einleitung}
\subsection{Beschreibung der Aufgabe}
Dieses Dokument dient als Projektbeschreibung für das Semesterprojekt des Moduls B42 Software Engineering 2 im Sommersemester 2023.
Die Aufgabe besteht darin, das Spiel "Superhirn" (Mastermind) zu implementieren, wobei Prinzipien, Methoden und Werkzeuge des Software Engineerings verwendet werden sollen.

\subsection{Teammitglieder}
\begin{description}
\item[Abderrahman Nofal] 582798
\item[Alexander Maier] 574287
\item[Leonas von Medem] 583776
\item[Pedro André Sanchez Valdivieso de Souza] 582986
\end{description}

\subsection{Gewählter Software-Prozess}
%TODO

\section{Analyse}
Nach Aufgabenstellung soll das Programm besonders folgende zwei funktionale Anforderungen erfüllen:

\begin{enumerate}
\item Sowohl als Codierer als auch als Rater agieren können;
\item Die gleichen Regeln des Originalspiels haben, diese sind:
\begin{enumerate}
\item Es gibt zwei Spieler: ein Codierer und ein Rater;
\item Der Codierer legt zu Beginn verdeckt einen vierstelligen geordneten Farbcode fest, der aus sechs Farben ausgewählt wird (wobei jede Farbe auch mehrmals verwendet werden kann);
\item Der Rater versucht, den Code herauszufinden
Dazu setzt er einen Farbcode als Frage.
\item Auf jeden Zug hin bekommt der Rater die Information,
\begin{enumerate}
\item wie viele Stifte er in Farbe und Position richtig gesetzt hat (roter Stift) und
\item wie viele Stifte zwar die richtige Farbe haben, aber an einer falschen Position stehen (weißer Stift); 
\end{enumerate}
\item Alle Fragen und Antworten bleiben bis zum Ende des Spiels gesteckt;
\item Das Spielfeld erlaubt max. 10 Rateversuche,
\end{enumerate}
\end{enumerate}


\subsection{Prozess}
\subsection{Ergebnisse}

\section{Design}
\subsection{Prozess}
\subsection{Entwickeltes Design}

\section{Implementierung}
%darunter zu schreiben: besonders interessante Implementierungsmerkmale
\subsection{TODO}
\subsection{TODO}
\subsection{TODO}

\section{Qualitätssicherung}
\subsection{Qualitätssicherungs-Prozess und -Strategie}
\subsubsection{Qualitätssicherungs-Maßnahmen}

\section{Fazit}
\subsection{Beurteilung des Ergebnisses}
\subsection{Beurteilung des Prozesses}
\subsection{Einschränkungen} %was hätte man anders/besser machen können?

\end{document}
\begin{document}
\[\]
\end{document}